\documentclass[a4paper]{book}
\usepackage[ngerman]{babel}
\usepackage{csquotes}
\usepackage{amsmath}
\usepackage{blkarray}

\MakeOuterQuote{"}

\title{Zusammenfassung Automatisierungstechnik}
\author{Jonas Otto}
\date{Sommersemester 2022}
\begin{document}
\maketitle
\tableofcontents

\chapter{Ereignisdiskrete Systeme}
\section{Zustandsbeschreibung von Petri-Netzen}

Der Zustand eines Petri-Netzes wird durch einen \emph{Markierungsvektor} $m$ angegeben.
Das ist bei uns ein binärer Vektor der angibt welche Stellen mit Marken belegt sind.

Ein Schaltvorgang wird mit einem \emph{Transitionsvektor} $t$ angegeben.
Der Transitionsvektor enthält einen $1$-Eintrag für die schaltende Transition.

Die Netzmatrix beschreibt Spaltenweise den Markenfluss beim Schalten der Transitionen:

\begin{equation}
    N =
    \begin{blockarray}{ccc}
        & t1 & t2 \\
        \begin{block}{c(cc)}
            s1 & -1 & -1 \\
            s2 &  1 &  0 \\
            s3 &  0 &  1 \\
        \end{block}
    \end{blockarray}
\end{equation}

Damit kann man dann die Zustandsgleichung aufstellen:
\begin{equation}
    m(k+1) = m(k) + N t(k)
\end{equation}

Mit der wichtigen Nebenbedingung
\begin{equation}
    t(k) \in \mathcal{T}_\text{akt}(m(k), u(k))
\end{equation}
welche fordert, dass die Transition aktiviert sein muss, also alle
Prästellen markiert und alle Poststellen unmarkiert sind, und die Transition
durch den Eingang $u(k)$ "freigeschaltet" ist.

\section{Analyse von Petri-Netzen}
\subsection{Zyklen: T-Invarianten}
Ein Zyklus kann so definiert werden:
\begin{equation}
    m(k_z) = m(0) + N \sum_{\kappa=0}^{k_z-1} t(\kappa) \overset{!}{=} m(0)
\end{equation}
Oder äquivalent
\begin{equation}
    N \sum_{\kappa=0}^{k_z-1} t(\kappa) = N i_T = 0
\end{equation}

Wenn jetzt also so ein passendes $i_T$ (ganzzahlig, $>0$) existiert,
nennt man das \emph{T-Invariante}.
Wenn zusätzlich die T-Invariante als gültige Abfolge von Transition darstellbar ist,
folgt daraus dass das Petri-Netz einen Zyklus besitzt.

\subsection{Lebendigkeit}
Eine \emph{tote Markierung} ist eine Markierung, in welcher keine Marken mehr fließen.
Wenn keine toten Markierungen auftreten, ist das Netz lebendig.

Jedes lebendige Petri-Netz besitzt mindestens einen Zyklus und damit eine T-Invariante.

\subsection{Konstante Markensumme: S-Invarianten}


\chapter{Multi-Agenten-Systeme}


\end{document}
