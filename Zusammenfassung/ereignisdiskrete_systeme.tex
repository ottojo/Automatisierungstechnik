\section{Zustandsbeschreibung von Petri-Netzen}

Der Zustand eines Petri-Netzes wird durch einen \emph{Markierungsvektor} $m \in \{0,1\}^{|S|}$ angegeben.
Das ist bei uns ein binärer Vektor der angibt welche Stellen mit Marken belegt sind.

Ein Schaltvorgang wird mit einem \emph{Transitionsvektor} $t \in \{0,1\}^{|S|}$ angegeben.
Der Transitionsvektor enthält einen $1$-Eintrag für die schaltende Transition.

\subsection{Netzmatrix}
Die Netzmatrix beschreibt Spaltenweise den Markenfluss beim Schalten der Transitionen:

\begin{equation}
    N =
    \begin{blockarray}{ccc}
        & t1 & t2 \\
        \begin{block}{c(cc)}
            s1 & -1 & -1 \\
            s2 &  1 &  0 \\
            s3 &  0 &  1 \\
        \end{block}
    \end{blockarray}
\end{equation}

\subsection{Zustandsgleichung}
Damit kann man dann die Zustandsgleichung aufstellen:
\begin{equation}
    m(k+1) = m(k) + N t(k)
\end{equation}

Mit der wichtigen Nebenbedingung
\begin{equation}
    t(k) \in \mathcal{T}_\text{akt}(m(k), u(k))
\end{equation}
welche fordert, dass die Transition aktiviert sein muss, also alle
Prästellen markiert und alle Poststellen unmarkiert sind, und die Transition
durch den Eingang $u(k)$ "freigeschaltet" ist.

Die Anfangsmarkierung ist $m(0)$.

\section{Analyse von Petri-Netzen}
\subsection{Zyklen: T-Invarianten}
Ein Zyklus kann so definiert werden:
\begin{equation}
    m(k_z) = m(0) + N \sum_{\kappa=0}^{k_z-1} t(\kappa) \overset{!}{=} m(0)
\end{equation}
Oder äquivalent
\begin{equation}
    N \sum_{\kappa=0}^{k_z-1} t(\kappa) = N i_T = 0
\end{equation}

Wenn jetzt also so ein passendes $i_T$ (ganzzahlig, $>0$) existiert,
nennt man das \emph{T-Invariante}.
Wenn zusätzlich die T-Invariante als gültige Abfolge von Transition darstellbar ist,
folgt daraus dass das Petri-Netz einen Zyklus besitzt.

\subsection{Lebendigkeit}
Eine \emph{tote Markierung} ist eine Markierung, in welcher keine Marken mehr fließen.
Wenn keine toten Markierungen auftreten, ist das Netz lebendig.

Jedes lebendige Petri-Netz besitzt mindestens einen Zyklus und damit eine T-Invariante.

\subsection{Konstante Markensumme: S-Invarianten}
Man fordert dass für jeden Zeitpunkt die gewichtete Markensumme konstant ist:
\begin{equation}
    \sum_{i=1}^{|S|} i_{S,i} m_i(k) = i_S^T m(k) = \text{const} \quad
    \forall k\in \mathbb{N}_0, i_{S,i}\in \mathbb{N}_0
\end{equation}
mit Gewichtsvektor $i_S \in \mathbb{N}_0^{|S|}$.

Induktiv bedeutet das
\begin{equation}
\begin{gathered}
    i_S^Tm(k+1) = i_S^Tm(k) + i_S^TNt(k) \stackrel{!}{=} i_S^Tm(k) \\
    \iff i_S^TN = 0
\end{gathered}
\end{equation}
wobei $i_S \in \mathbb{N}_0^{|S|}$ die nichttriviale Lösung sein soll.
Dann ist $i_S$ eine \emph{S-Invariante} des Petri-Netzes.

Wenn eine S-Invariante keinen 0-Eintrag enthält, bedeutet das, dass sich die
Anzahl der Markierungen im Petri-Netz nicht ändern kann.

\section{Petri-Netz-Regler}


Die Spezifikation des Reglers wird mit einer Ungleichungsbedingung
formuliert:
\begin{equation}
    L m(k) \leq b, \quad k\in\mathbb{N}_0, \quad L\in\mathbb{N}_0^{n_R\times |S|}, \quad  b\in\mathbb{N}_0^{n_R}
\end{equation}
mit $n_R$ der Anzahl an Bedingungen (Ordnung des Reglers),
und $\leq$ Elementweise.

Der ereignisdiskrete Regler ist selbst ein Petri-Netz,
mit Zustandsdarstellung
\begin{equation}
    m_R(k+1) = m_R(k)+N_Rt(K), \quad m_R \in \mathbb{N}_0^{n_R}
\end{equation}
, das subscript $_R$ steht für "Regler".

Achtung: $t(k)$ ist auch die Transition der Strecke!\\
Es muss gelten $N_R \in \mathbb{Z}^{n_R \times |\mathcal{T}|}$
mit $|\mathcal{T}|$ der Anzahl an Transitionen der Strecke.

Zum Reglerentwurf wird die Ungleichungsbedingung auf S-Invarianten-Vorgabe
zurückgeführt.

Man setzt
\begin{equation}
    Lm(k) + m_R(k) = b
\end{equation}
(das geht, weil $m_R(k) \in \mathbb{N}_0^{n_R}$),
und dasst Regler und Strecke zusammen:
\begin{equation}
    \begin{bmatrix}
        m(k+1)\\m_R(k+1)
    \end{bmatrix}
    =
    \begin{bmatrix}
        m(k)\\m_R(k)
    \end{bmatrix}
    +
    \begin{bmatrix}
       N\\N_R 
    \end{bmatrix}
    t(k)
\end{equation}
Was dann erlaubt die Bedingung umzuschreiben zu
\begin{equation}
    \begin{bmatrix}
        L&I
    \end{bmatrix}
    \begin{bmatrix}
        m(k)\\r_R(k)
    \end{bmatrix}
    =b
\end{equation}
Was bedeutet dass die Zeilen von $\begin{bmatrix}
    L&I
\end{bmatrix}$ gerade S-Invarianten des geregelten Systems seien müssen.

Für $k=0$ ergibt sich durch Einsetzen
\begin{equation}
    m_R(0) = b-Lm(0)
\end{equation}
(Was dann auch $b$ garantiert, ohne die Anfangsbedingung würden die S-Invarianten
zwar konstantes, aber beliebiges $b$ fordern.)

Um nun $N_R$ zu konstruieren wird die Definition der S-Invarianten genutzt
(für jede Zeile, rechts steht jetzt ein Nullvektor statt Skalar):
\begin{equation}
    \begin{gathered}
        \begin{bmatrix}
            L&I
        \end{bmatrix}
        \begin{bmatrix}
            N\\N_R
        \end{bmatrix}
        =0\\
        \iff LN + N_R = 0 \\
        \iff N_R = -LN
    \end{gathered}
\end{equation}

