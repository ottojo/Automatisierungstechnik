\section{Graphentheorie}
\paragraph{Gradmatrix}
Die Gradmatrix $D_\mathcal{G}$ enthält auf der Hauptdiagonalen den Eingangsgrad des jeweiligen Knotens.

\paragraph{Adjazenzmatrix}
Die Adjazenzmatrix $A_\mathcal{G}$ enthält im Element $a_{ij}$
das Kantengewicht der Kante $(j \rightarrow i)$.

\paragraph{Laplacematrix}
\begin{equation}
    \tag{Laplacematrix}
    L_\mathcal{G} = D_\mathcal{G} - A_\mathcal{G}
    \label{eqn:laplace_matrix}
\end{equation}
Die Laplacematrix berechnet sich aus Gradmatrix $D$ und
Adjazentmatrix $A$.

\section{Leader-Follower Zustandssynchronisierung}
\paragraph{Leader-Follower-Matrix}
\begin{equation}
    \tag{Leader-Follower-Matrix}
    H = L_\mathcal{G} + L_0 \in \mathbb{R}^{N\times N}
    \label{eqn:lf_matrix}
\end{equation}
$L_0$ ist dabei die Diagonalmatrix mit $L_{0,ii}=a_{i0}$.

\subsection{Entwurf Netzwerkregler}

Ziel ist ein Netzwerkregler der Form
\begin{equation}
    u_i = -\sum_{j=1}^N a_{ij} (Kx_i - Kx_j) - a_{i0} (Kx_i - Kx_0)
\end{equation}
Die Rückführmatrix $K$ bestimmt sich durch
\begin{equation}
    K=c R^{-1} B^T P
\end{equation}
mit $P$ Lösung der algebraischen Riccati-Gleichung
\begin{equation}
    \tag{ARE}
    A^T P + P A - P B R^{-1} B^T P + Q = 0
\end{equation}
wobei $Q \in \mathbb{R}^{n\times n}$ und $R \in \mathbb{R}^{n\times n}$
positiv definit, mit $n$ Dimension des Zustandes und $p$ Dimension des
Inputs.

Für den Parameter $c$ muss gelten
\begin{equation}
    c \geq \frac{1}{2 \cdot \min_{i=1,\ldots,N} \operatorname{Re} \lambda_i(H)}
\end{equation}

\todo[inline]{Fertig machen (Blatt3)?}

\subsubsection{Sonderfall: Reduzible Kommunikationstopologie}
Wenn die Leader-Follower-Matrix als untere ($r\times r$) Blockdiagonalmatrix
dargestellt werden kann (evtl nach Ändern der Nummerierung der Agenten),
können $r$ unabhängige Stabilisierungsprobleme mit $r$ unterschiedlichen
Rückführmatrizen gelöst werden.

\subsubsection{Sonderfall: Kreisfreie Kommunikationstopologie}
Eine kreisfreie Topologie erlaubt umnummerierung sodass die
Leader-Follower-Matrix eine untere ($N\times N$) Dreiecksmatrix ist.

Dann können $N$ unterschiedliche Rückführmatrizen bestimmt werden.
\todo[inline]{Kreisfreie Kommunikationstopologie, Eigenwertvorgabe: Blatt 3f}

\subsection{Analyse Netzwerkregler}
\todo[inline]{Synchronisierungs-Eigenwerte: Blatt 3g}

\section{Leaderless Zustandssynchronisierung}
\paragraph{Folgefehlerdynamik}
Es gibt zwar keinen Führungsagenten, aber wir können einen Folgefehler
bezüglich eines ausgewählten Agentens ($x_1$) definieren.
Damit ergeben sich die \emph{Synchronisierungskoordinaten}:
\begin{equation}
    \tilde{x} = \begin{bmatrix}
        x_1 \\
        \varepsilon
    \end{bmatrix}
    = \begin{bmatrix}
        x_1 \\
        x_2 - x_1 \\
        \vdots \\
        x_N - x_1
    \end{bmatrix}
    = \left(\begin{bmatrix}
        1 & 0 \\
        -1_{N-1} & I_{N-1}
    \end{bmatrix}
    \otimes I_n
    \right) x
    = (\Theta \otimes I_n)x
\end{equation}
(Die Einträge $1$, $0$, und $-1_{N-1}$ in $\Theta$ sind hierbei als skalar bzw Zeilen- und Spaltenvektor zu verstehen.)

Aus der Systemdynamik
\begin{equation}
    \dot{x} = (I_N \otimes A) x + (I_N \otimes B)u
\end{equation}
folgt dann nach Transformation auf Synchronisierungskoordinaten die
\emph{Folgefehlerdynamik}:
\begin{equation}
    \dot{\tilde{x}} = (I_N \otimes A) \tilde{x} + (\Theta \otimes B)u
\end{equation}

\section{Leaderless Ausgangssynchronisierung}
