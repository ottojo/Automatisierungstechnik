\section{Graphentheorie}
\paragraph{Gradmatrix}
Die Gradmatrix $D_\mathcal{G}$ enthält auf der Hauptdiagonalen den Eingangsgrad des jeweiligen Knotens.

\paragraph{Adjazenzmatrix}
Die Adjazenzmatrix $A_\mathcal{G}$ enthält im Element $a_{ij}$
das Kantengewicht der Kante $(j \rightarrow i)$.

Die Adjazenzmatrix eines nicht stark zusammenhängenden Graphen ist reduzibel.
Eine Matrix ist reduzibel, wenn mittels Permutationsmatrix $P$ gilt
\begin{equation}
    PMP^T = \begin{bmatrix}
        M_{11} & 0 \\
        M_{12} & M_{22}
    \end{bmatrix}
\end{equation}
mit quadratischem $M_{11}$.
Permutationsmatrix bedeutet dass $PM$ Zeilen tauscht, und $MP^T$ Spalten tauscht.

Wenn $PA_\mathcal{G}P^T$ eine untere Blockdreiecksmatrix ist mit irreduziblen
Diagonalblockmatrizen, entspricht das der \emph{Frobeniusform}.
Die Diagonalblockmatrizen beschrieben dann stark zusammenhängende Teilgraphen.


\paragraph{Laplacematrix}
\begin{equation}
    \tag{Laplacematrix}
    L_\mathcal{G} = D_\mathcal{G} - A_\mathcal{G}
    \label{eqn:laplace_matrix}
\end{equation}
Die Laplacematrix berechnet sich aus Gradmatrix $D$ und
Adjazenzmatrix $A$.

\begin{itemize}
    \item Ein Eigenwert von $L_\mathcal{G}$ bei 0 ist einfach $\iff$ Graph zusammenhängend
    \item $\operatorname{rang} L_\mathcal{G} = N-1$ ($N$ Knoten) $\iff$ Graph zusammenhängend
    \item Die Aussagen bzgl. Reduktion der Adjazenzmatrix gelten auch für die Laplacematrix.
    \item Das Laplacespektrum ist die Menge der Eigenwerte von $L_\mathcal{G}$.
\end{itemize}

\paragraph{Inzidenzmatrix}
Elemente $n_{ij}$ der Inzidenzmatrix $N_\mathcal{G} \in \{-1,0,1\}^{N \times M}$ eines Graphen mit $N$ Knoten und $M$ Kanten:
\begin{equation}
    n_{ij} = \begin{cases}
        -1, & \text{wenn Kante $j$ im Knoten $i$ startet} \\
        1,  & \text{wenn Kante $j$ im Knoten $i$ endet} \\
        0,  & \text{sonst}
    \end{cases}
\end{equation}
Jede Spalte stellt dabei eine Kante dar.

\paragraph{Zusammenhängend}
Ein Wurzelknoten existiert.
\begin{itemize}
    \item $\iff$ Adjazenzmatrix $A_\mathcal{G}$ irreduzibel.
    \item $\iff$ Ein Eigenwert von $L_\mathcal{G}$ bei 0 ist einfach
\end{itemize}


\paragraph{Stark Zusammenhängend}
Jeder Knoten ist Wurzelknoten, es gibt von jedem Knoten einen Pfad zu
jedem anderen Knoten.
$\iff$ Laplacematrix $L_\mathcal{G}$ irreduzibel.


\paragraph{Vollständig Zusammenhängend}
= fully connected.


\section{Systemtheorie}

\subsection{Steuerbarkeit}
\label{sec:steuerbar}
Der Eigentwert $\lambda_i$ von $A$ ist steuerbar wenn gilt
\begin{equation}
    \operatorname{rang}\begin{bmatrix}
        A-\lambda_i I & B
    \end{bmatrix} = n
\end{equation}
Wobei $A \in \mathbb{R}^{n\times n}$.
Sind alle Eigenwerte in der rechten Halbebene steuerbar, ist $(A, B)$
steuerbar bzw stabilisierbar.

\subsection{Schur Zerlegung}
Quadratische Matrix $A$:
\begin{equation}
    R=U^H A U = 
\end{equation}
\begin{itemize}
    \item R ist obere Dreiecksmatrix ($\implies$ EW auf Hauptdiagonalen)
    \item Die Matrizen $A$ und $R$ sind Ähnlich, haben also die gleichen Eigenwerte
\end{itemize}

Mit Blockmatrizen/Kronecker-Produkten funktioniert das auch.

\subsection{Signalmodelle}
\subsubsection{Konstant}
\begin{align}
    \dot{v} &= 0 \\
    v(t) &= v(0) = c \\
    \sigma &= \{0\}
\end{align}

\subsubsection{Rampe}
\begin{align}
    \dot{v} &= \begin{bmatrix}
        0 & 1 \\
        0 & 0
    \end{bmatrix} v \\
    v(0) &= \begin{bmatrix}
        a \\ b
    \end{bmatrix} \\
    v(t) &= \begin{bmatrix}
        b \cdot t + a \\
        b
    \end{bmatrix} \\
    \sigma &= \{0, 0\}
\end{align}

\subsubsection{Sinus}
\begin{align}
    \dot{v} &= \begin{bmatrix}
        0 & \omega \\
        -\omega & 0
    \end{bmatrix} v \\
    v(0) &= \begin{bmatrix}
        \sin(\varphi_1) \\
        \cos(\varphi_2)
    \end{bmatrix} \\
    v(t) &= \begin{bmatrix}
        \sin(\omega t + \varphi_1) \\
        \cos(\omega t + \varphi_2)
    \end{bmatrix} \\
    \sigma &= \{+j\omega, -j\omega\}
\end{align}

\subsection{Sonstige Sinnvolle Rechenregeln}
\begin{itemize}
    \item $\det(\text{Block-Diag-Matrix}) = \prod_i \det(\text{Block}_i)$
\end{itemize}

\section{Leader-Follower Zustandssynchronisierung}
\paragraph{Leader-Follower-Matrix}
\label{par:leader_follower_matrix}
\begin{equation}
    \tag{Leader-Follower-Matrix}
    H = L_\mathcal{G} + L_0 \in \mathbb{R}^{N\times N}
    \label{eqn:lf_matrix}
\end{equation}
$L_0$ ist dabei die Diagonalmatrix mit $L_{0,ii}=a_{i0}$.
$L_\mathcal{G}$ enthält den Führungsagenten nicht!

Es gilt $\det(H) \neq 0$ und $\operatorname{Re}\lambda_i(H) > 0$,
wenn der erweiterte Graph einen Spannbaum mit dem Führungsagenten als
Wurzel besitzt.

\subsection{Folgefehler}
Der Folgefehler ist definiert als:
\begin{equation}
    \varepsilon = \begin{bmatrix}
        x_1-x_0 \\
        \vdots \\
        x_N-x_0
    \end{bmatrix}
    = x-1_N \otimes x_0
\end{equation}
wobei $x_i$ der Zustandsvektor des Agenten $i$ ist.
Mit ein bisschen einsetzen von $\dot{x} = (I_N \otimes A)x + (I_N \otimes B)u$
und Umformen kommt man daraus auf die \emph{Folgefehlerdynamik}
\begin{equation}
    \dot{\varepsilon} = (I_N \otimes A)\varepsilon + (I_N \otimes B)u
\end{equation}

Mit einem Regler $u=-(I_N \otimes K)\varepsilon$  ergibt sich das
\emph{geregelte Folgefehlersystem}
\begin{equation}
    \dot{\varepsilon} = (I_N \otimes (A-BK))\varepsilon
\end{equation}

Aufgabe ist jetzt $K$ so zu bestimmen dass das geregelte Fehlersystem unseren
Anforderungen (exponentielle Stabilität) entspricht.

Dies beachtet allerdings noch nicht die Netzwerktopologie.
Das macht im nächsten Abschnitt die

\subsection{Diffusive Kopplung}
Im Hinblick auf die Netzwerktopologie findet sich der Folgefehler in der
diffusiven Kopplung wieder:

\begin{align}
    \bar{u}_i &= \sum_{j=1}^N a_{ij} (x_i-x_j) + a_{i0}(x_i-x_0) \\
    &= \sum_{j=1}^N a_{ij} (\varepsilon_i - \varepsilon_j) + a_{i0} \varepsilon_i \\
    &= \left(e_i^T\underbrace{(L_G+L_0)}_H \otimes I_n\right)\varepsilon
\end{align}
Wobei $e_i^T$ ein Zeilenvektor ist mit $1$ an Stelle $i$, sonst $0$.

Aggregieren der $\bar{u}_i$ Elemente ergibt
\begin{equation}
    \bar{u} = (H \otimes I_n)\varepsilon
\end{equation}

\subsection{Fehlersystem mit diffusiver Kopplung}
Nun lässt sich eine Rückführung definieren:
\begin{equation}
    u=-\tilde{K}(H\otimes I_n)\varepsilon
\end{equation}
wobei $\tilde{K}$ eine Blockmatrix ist mit den Rückführmatrizen $K_i$ der einzelnen
Agenten auf der Hauptdiagonalen.

Das geregelte Fehlersystem ist nun
\begin{equation}
    \dot{\varepsilon} = (I_N \otimes A - (I_N \otimes B) \tilde{K}(H \otimes I_n))\varepsilon
    \label{eqn:fehlersystem_geregelt_ktilde}
\end{equation}

\subsection{Simultanes Stabilisierungsproblem}
Im folgenden wird für jeden Folgeagenten die Gleiche Rückführmatrix $K$ verwendet.
Das bedeutet, dass diese Matrix \emph{simultan} jeden Folgeagenten stabilisieren
muss:
\begin{equation}
    \tilde{K}=I_N \otimes K
\end{equation}
Das Stellgesetz lautet nun
\begin{equation}
    u=-(I_N \otimes K) (H \otimes I_n)\varepsilon = -(H \otimes K)\varepsilon
\end{equation}

Mit dem Ziel, die leader-follower Matrix $H$ auf Schursche Normalform zu bringen,
wird die Transformation
\begin{align}
    \tilde{\varepsilon} &= (T^H \otimes I_n)\varepsilon \\
    \varepsilon &= (T \otimes I_n)\tilde{\varepsilon}
\end{align}
eingeführt.

Eingesetzt in die Gleichung des geregelten Fehlersystems \ref{eqn:fehlersystem_geregelt_ktilde}
ergibt sich:
\begin{align}
    \dot{\tilde{\varepsilon}} &=((T^H \otimes I_n)(I_N \otimes A)(T \otimes I_n) - (T^H \otimes I_n)(I_N \otimes B)\tilde{K}(H \otimes I_n)(T \otimes I_n)) \tilde{\varepsilon} \\
    &= (I_N \otimes A - H_\urcorner \otimes BK)\tilde{\varepsilon} \\
    &= \tilde{A}\tilde{\varepsilon} 
\end{align}
mit $H_\urcorner = T^H H T$.

Da die Diagonalelemente von $H_\urcorner$ gerade dessen Eigenwerte, und damit auch die
Eigenwerte von $H$ sind (Matrizen sind Ähnlich), lassen sich die Blockdiagonalelemente von
$\tilde{A}$ darstellen als:
\begin{equation}
    \tilde{A}_{ii} = A-\lambda(H)BK
\end{equation}

$K$ muss nun also so bestimmt werden, dass diese $N$ unterschiedlichen Matrizen simultan
zu Hurwitz-Matrizen ($\operatorname{Re}\lambda < 0)$ macht.

Notwendige Bedingung dafür ist $\det H \neq 0$ (Siehe \ref{par:leader_follower_matrix},
das kann man evtl am Graph sehen).

Der Riccati-Entwurf kann die simultane Stabilisierung in diesem fall immer lösen,
was das ganze zur hinreichenden Bedingung macht.

\subsection{Voraussetzungen Simultane Stabilisierung}
\begin{itemize}
    \item $(A,b)$ der Agenten muss stabilisierbar sein: Siehe \ref{sec:steuerbar}
    \item Führungsagent ist Wurzelknoten des erweiterten Graphens:
          Angeben eines Spannbaums oder zeigen dass $\det H \neq 0$
          für die Leader-Follower Matrix.
\end{itemize}

\subsection{Zustandstransformation}
Wenn z.B. der Abstand von Agenten untereinander geregelt wird, bietet sich eine
Zustandstransformation an die das berücksichtigt. Wenn in der ersten Komponente
z.B. die Position steht und jeder Agent $i$ den Abstand $\Delta x_i$ haben soll:
\begin{equation}
    \bar{x}_i = x_i - \begin{bmatrix}
        \Delta x_i \\ 0
    \end{bmatrix}
\end{equation}
Die $A$ Matrix etc muss dann für den neuen Zustand bestimmt werden, und im Folgenden
wird damit dann gerechnet.


\subsection{Entwurf Netzwerkregler}
\label{sec:lf_nw_regler_entwurf}
Ziel ist ein Netzwerkregler der Form
\begin{align}
    u_i &= -\sum_{j=1}^N a_{ij} (Kx_i - Kx_j) - a_{i0} (Kx_i - Kx_0) \\
        &= -K(e_i^T H \otimes I_n)\varepsilon
\end{align}
Die Rückführmatrix $K$ bestimmt sich durch
\begin{equation}
    K=c R^{-1} B^T P
\end{equation}
mit $P \in \mathbb{R}^{n\times n}$ (symmetrisch) Lösung der algebraischen Riccati-Gleichung
\begin{equation}
    \tag{ARE}
    A^T P + P A - P B R^{-1} B^T P + Q = 0
\end{equation}
wobei $Q \in \mathbb{R}^{n\times n}$ und $R \in \mathbb{R}^{p\times p}$
positiv definit, mit $n$ Dimension des Zustandes und $p$ Dimension des
Inputs.

Für den Parameter $c$ muss gelten
\begin{equation}
    c \geq \frac{1}{2 \cdot \min_{i=1,\ldots,N} \operatorname{Re} \lambda_i(H)}
\end{equation}

\subsection{Sonderfälle}
\subsubsection{Sonderfall: Reduzible Kommunikationstopologie}
Wenn die Leader-Follower-Matrix als untere ($r\times r$) Blockdiagonalmatrix
dargestellt werden kann (evtl nach Ändern der Nummerierung der Agenten),
können $r$ unabhängige Stabilisierungsprobleme mit $r$ unterschiedlichen
Rückführmatrizen gelöst werden.

Dabei ist zu beachten dass sich im Vergleich zum allgemeinen Fall in der
Gleichung für $K$ nur $c$ ändert, $P$ und $B$ sind nicht vom Graph abhängig.

\subsubsection{Sonderfall: Kreisfreie Kommunikationstopologie}
Eine kreisfreie Topologie erlaubt umnummerierung sodass die
Leader-Follower-Matrix eine untere ($N\times N$) Dreiecksmatrix ist.

Dann können $N$ unterschiedliche Rückführmatrizen bestimmt werden:

Das Fehlersystem ist jetzt
\begin{align}
    \dot{\varepsilon} &= (I_N \otimes A - (I_N \otimes B)\tilde{K}(H \otimes I_n))\varepsilon \\
    &= \begin{bmatrix}
        A-h_{11}BK_1 & & \\
        \vdots & \ddots & \\
        -h_N B K_N & \dots & A-H_{NN} BK_N
    \end{bmatrix} \varepsilon
\end{align}.

Daraus folgt, dass die Stabilisierung mit unterschiedlichen Rückführmatrizen $K_i$
gelöst werden kann, indem jeweils $(A-h_{ii}BK_i)$ stabilisiert wird.
Das haben wir in der Übung mit Eigenwertvorgabe (Koeffizientenvergleich) gemacht.

\section{Leaderless Zustandssynchronisierung}
\paragraph{Folgefehlerdynamik}
Es gibt zwar keinen Führungsagenten, aber wir können einen Folgefehler
bezüglich eines ausgewählten Agentens ($x_1$) definieren.
Damit ergeben sich die \emph{Synchronisierungskoordinaten}:
\begin{equation}
    \tilde{x} = \begin{bmatrix}
        x_1 \\
        \varepsilon
    \end{bmatrix}
    = \begin{bmatrix}
        x_1 \\
        x_2 - x_1 \\
        \vdots \\
        x_N - x_1
    \end{bmatrix}
    = \left(\begin{bmatrix}
        1 & 0 \\
        -1_{N-1} & I_{N-1}
    \end{bmatrix}
    \otimes I_n
    \right) x
    = (\Theta \otimes I_n)x
\end{equation}
(Die Einträge $1$, $0$, und $-1_{N-1}$ in $\Theta$ sind hierbei als skalar bzw Zeilen- und Spaltenvektor zu verstehen.)

\begin{equation}
    \Theta^{-1} = \begin{bmatrix}
        1 & 0 \\
        1_{N-1} & I_{N-1}
    \end{bmatrix}    
\end{equation}

Aus der Systemdynamik
\begin{equation}
    \dot{x} = (I_N \otimes A) x + (I_N \otimes B)u
\end{equation}
folgt dann nach Transformation auf Synchronisierungskoordinaten die
\emph{Folgefehlerdynamik}:
\begin{equation}
    \dot{\tilde{x}} = (I_N \otimes A) \tilde{x} + (\Theta \otimes B)u
\end{equation}

\paragraph{Stellgesetz}
\begin{equation}
    u_i = -K \sum_{j=1}^N a_{ij}(x_i-x_j)
\end{equation}

\subsection{Synchronisierungstrajektorie}
Der Netzwerkregler stellt sicher dass die Agenten gegen die Synchronisierungstrajektorie
konvergieren (auch "virtueller Führungsagent"):
\begin{align}
    \dot{x}_s &= Ax_s \\
    x_s(0) &= (l^T \otimes I_n) x(0)
\end{align}
Der Vektor $l \in \mathbb{R}^N$ gewichtet dabei die Anfangszustände der einzelnen
Agenten.
Zur Bestimmung genügen die constraints 
\begin{align}
    l^T 1_N &= 1 \\
    l^T L_\mathcal{G} &= 0
\end{align}
Dies folgt von der Herleitung über die \emph{invariante Trajektorie}, für die
Unabhängigkeit von $K$ gefordert wird.

Nur \emph{Wurzelagenten} können einen Beitrag zur 
Synchronisierungstrajektorie leisten!

\subsection{Entwurf Netzwerkregler}

\subsubsection{Bedingung}
Damit sich eine Rückführmatrix $K$ bestimmen lässt, muss gelten
\begin{equation}
    \operatorname{Re} \lambda_i(L_{22}) > 0
\end{equation}
bzw.
\begin{equation}
    \det L_{22} \neq 0
\end{equation}
wobei $L_{22}$ folgt aus
\begin{equation}
    \Theta L_\mathcal{G} \Theta^{-1} = \begin{bmatrix}
        0 & \Theta L_\mathcal{G}M
    \end{bmatrix}
    = \begin{bmatrix}
        0 & l_{12}^T \\
        0_{N-1} & L_{22}
    \end{bmatrix}
\end{equation}

$\det L_{22} \neq 0$ folgt auch direkt daraus, dass der Graph zusammenhängend ist.

Außerdem muss $(A,b)$ steuerbar und stabilisierbar sein.

\subsubsection{Entwurf}
Bestimmung von $K$ wie in \ref{sec:lf_nw_regler_entwurf},
aber mit $L_{22}$ statt H.


\pagebreak
\section{Leader-Follower Ausgangssynchronisierung}
\subsection{Kooperativ angeregtes Internes Modell}
Das kann man so machen wie Leaderless mit internem Modell
(\ref{sec:leaderless_ausgang_internes_modell}).
Beim Riccati Entwurf muss für die Berechnung $H$ statt $L_{22}$ verwendet
werden.

\subsection{Führungsbeobachter}
In \ref{sec:koop_beob} ist angegeben wie das mit heterogenen Agenten funktioniert,
das geht natürlich auch hier.

\pagebreak
\section{Leaderless Ausgangssynchronisierung}
\label{sec:leaderless_ausgang}
\subsection{Kooperativ angeregtes Internes Modell}
\label{sec:leaderless_ausgang_internes_modell}
Hier gibt es statt Führungsagent ein Führungsmodell:
\begin{align}
    \dot{v}_w &= S_w v_w \\
    w &= Q_w v_w
\end{align}

Und bei Bedarf Störmodelle $S_{d_i}$.

Der Netzwerkregler für die Ausgangssynchronisierung beinhaltet die
Dynamik des Führungsmodells und aller Störmodelle.
Dafür baut man das kooperativ angeregtes interne Modell:
\begin{equation}
    \dot{\bar{v}}_i = S^*\bar{v}_i + B_y \sum_{j=1}^N a_{ij} (y_i-y_j)
\end{equation}

\begin{equation}
    S^* = I_p \otimes S
\end{equation}
$S^*$ ist die $p=\dim y$ - fache Kopie von S, wobei S die Zusammenfassung des
Führungssignalmodells und Störmodells ist. Der Zyklische Anteil bedeutet, dass jede
Signalform nur ein mal enthalten sein soll.

\begin{equation}
    S = \operatorname{zykl}(\operatorname{diag} (S_w, S_{d1}, \dots, S_{dN}))
\end{equation}

\underline{$B_y = \operatorname{diag}(b_1,\dots,b_p)$} mit Spaltenvektoren $b_i$
muss weiterhin so gewählt werden dass \underline{$(S, b_i)$ steuerbar} ist.
Für Sinus- und Rampensignale kann üblicherweise $\begin{bmatrix}
    0 & 1
\end{bmatrix}^T$ gewählt werden.

Die \emph{Voraussetzungen} für erfolgreiche Ausgangssynchronisierung
mit internem-Modell-Prinzip sind:

\begin{itemize}
    \item Graph $\mathcal{G}$ zusammenhängend
    \item $(A,B)$ stabilisierbar
    \item Kein EW von $S$ ist invariante Nullstelle der Agenten
\end{itemize}

Invariante Nullstellen $\eta$ werden ermittelt durch
\begin{equation}
    \det \begin{bmatrix}
        A-\eta I & B \\
        C & 0
    \end{bmatrix} = 0
\end{equation}
Das System kann auch keine oder mehrere invariante Nullstellen haben.

Für die Stabilisierung setzt man das \emph{Stellgesetz} an in der Form
\begin{align}
    u&=\underbrace{-(I_N \otimes R_{\bar{v}})\bar{v}}_{\text{Lokal: }u^l} - \underbrace{(L_{\mathcal{G}} \otimes K_x)x}_{\text{Kooperativ: }u^k} \\
    u_i &= -R_{\bar{v}} \bar{v}_i - \sum_{j=1}^N a_{ij}K_x (x_i -x_j)
\end{align}

Eingesetzt und aggregiert ergibt das folgende \emph{Dynamik für das geregelte MAS} und das
\emph{kooperative interne Modell}:
\begin{align}
    \dot{x} &= (I_N \otimes A - L_{\mathcal{G}} \otimes BK_x) x - (I_N \otimes BR_{\bar{v}})\bar{v} \\
    \dot{\bar{v}} &= (I_N \otimes S^*)\bar{v} + (L_{\mathcal{G}}\otimes B_y C)x
\end{align}

Die Transformation auf Synchronisierungskoordinaten
\begin{equation}
    \tilde{x}=\begin{bmatrix}
        x_1 \\
        \varepsilon_x
    \end{bmatrix}
    = \begin{bmatrix}
        x_1 \\ x_2-x_1 \\ \vdots \\ x_N-x_1
    \end{bmatrix}
\end{equation}
erfolgt mittels $\tilde{x}=(\Theta \otimes I_n)x$, analog für $\tilde{v}$.


Das System kann man nun aufteilen in einen Anteil der instabil bleibt, und die
\emph{Synchronisierungsdynamik} vorgibt:
\begin{align}
    \dot{x}_1 &= Ax_1 - (l_{12}^T \otimes BK_x)\varepsilon_x - BR_{\bar{v}}\bar{v}_1 \\
    \dot{\bar{v}}_1 &= S^*\bar{v}_1 + (l_{12}^T \otimes B_y C)\varepsilon_x
\end{align}

Und die Fehlerzustände, welche mittels simultaner Stabilisierung
zu null geregelt werden sollen:
\begin{align}
    \dot{\varepsilon}_x &= (I_{N-1} \otimes A - L_{22} \otimes BK_x) \varepsilon_x - (I_{N-1} \otimes BR_{\bar{v}}) \varepsilon_{\bar{v}} \\
    \dot{\varepsilon}_{\bar{v}} &= (I_{N-1} \otimes S^*) \varepsilon_{\bar{v}} + (L_{22} \otimes B_y C) \varepsilon_x
\end{align}

Das simultane Stabilisierungsproblem ist dann
\begin{equation}
    \check{A}_{\urcorner, ii} = \tilde{A} - \lambda_i(L_{22}) \tilde{B} \begin{bmatrix}
        K_x & R_{\bar{v}}
    \end{bmatrix}
\end{equation}
Wobei $\check{A}_{\urcorner, ii}$ Hurwitz-Matrix (alle EW links der Imaginärachse)
sein soll für alle $\lambda_i(L_{22})$, mit
\begin{align}
    \tilde{A} &= \begin{bmatrix}
        A & 0 \\
        B_y C & S^*
    \end{bmatrix} \\
    \tilde{B} &= \begin{bmatrix}
        B \\ 0
    \end{bmatrix}
\end{align}

Das kann man mit ARE-Entwurf machen für das $(\varepsilon_x, \varepsilon_{\bar{v}})$-System:
\begin{equation}
    \boxed{
    \begin{bmatrix}
        K_x & R_{\bar{v}}
    \end{bmatrix}
    = cR^{-1}\tilde{B}^TP
    }
\end{equation}
Mit $P$ (symmetrisch) Lösung der ARE
\begin{equation}
    \tilde{A}^TP + P\tilde{A}-P\tilde{B}R^{-1}\tilde{B}^TP+Q = 0
\end{equation}
und
\begin{equation}
    c \geq \frac{1}{2 \min (\operatorname{Re} \lambda_i(L_{22}))}
\end{equation}


Die Synchronisierungstrajektorie ergibt sich dann als
\begin{equation}
    \begin{bmatrix}
        \dot{x}_s \\ \dot{\bar{v}}_s
    \end{bmatrix}
    =
    \begin{bmatrix}
        A & -BR_{\bar{v}} \\
        0 & S^*
    \end{bmatrix}
    \begin{bmatrix}
        x_s \\ \bar{v}_s
    \end{bmatrix}
\end{equation}

Da in $\dot{x}_s$ die (eventuell instabile) Agentendynamik $A$ enthalten ist, kann es
sinnvoll sein eine lokale Stabilisierung zu machen:

\begin{align}
    u_i &= -K_l x_i + \bar{u}_i \\
    \dot{x}_i &= \underbrace{(A-BK_l)}_{\text{Neue Dynamikmatrix }\tilde{A}_l}x_i + B\bar{u}_i
\end{align}

Was dann in der Synchronisierungstrajektorie
\begin{equation}
    \begin{bmatrix}
        \dot{x}_s \\ \dot{\bar{v}}_s
    \end{bmatrix}
    =
    \begin{bmatrix}
        \tilde{A}_l & -BR_{\bar{v}} \\
        0 & S^*
    \end{bmatrix}
    \begin{bmatrix}
        x_s \\ \bar{v}_s
    \end{bmatrix}
\end{equation}
resultiert. Das $\tilde{A}_l$ muss aber auch im Entwurf berücksichtigt werden!

\subsection{Führungsbeobachter}
Siehe \ref{sec:koop_beob}.

\pagebreak
\section{Synchronisierung heterogener Multiagentensysteme}

Um das heterogene Multiagentensystem zu synchronisieren, müssen die Agenten das \emph{kooperative Interne-Modell-Prinzip} erfüllen:

Gemeinsame Teildynamik $A_s$ mit Ausgang $C_s$.
\begin{equation}
    T_i A_i T_i^{-1} = \begin{bmatrix}
        A_s & * \\
        0 & *
    \end{bmatrix}
\end{equation}

\begin{equation}
    C_i T_i^{-1} = \begin{bmatrix}
        C_s & *
    \end{bmatrix}
\end{equation}

Bemerkung: Notwendig (aber nicht hinreichend) dafür ist, dass die Agenten gleiche Eigenwerte haben.

Wenn das nicht erfüllt ist, gibt es zwei Lösungsmöglichkeiten:
\begin{enumerate}
    \item Dynamik mit Regler mitberücksichtigen (kooperativ angeregtes internes Modell)
    \item Kooperative Führungsbeobachter
\end{enumerate}


\subsection{Kooperative Führungsbeobachter: Leader-Follower}
\label{sec:koop_beob}

Heterogene Agenten:
\begin{align}
    \dot{x}_i &= A_i x_i + B_i u_i \\
    y_i &= C_i x_i
\end{align}

Die Synchronisierungstrajektorie wird vom Führungsagenten vorgegeben, beschrieben
durch Signalmodell:
\begin{align}
    \dot{v}_0 &= S_w v_0 \\
    w &= Q_w v_0
\end{align}
(Anforderung: $(Q_w, S_w)$ beobachtbar)

Daraus resultiert das Führungssignal $w$.
Diesem Signal sollen die Ausgänge der Agenten folgen.

Dafür werden an den Agenten Führungsbeobachter entworfen, die den Zustand des
Führungsagenten rekonstruieren ($\hat{v}_i$).

Dann wird eine Führungsgrößenaufschaltung konstruiert, die eine Zustandsrückführung
der Agentenzustände und eine Aufschaltung des rekonstruieren
Führungsagentenzustands enthält.


\subsubsection{Informierte Agenten}

Bei informierten Agenten ist das ein klassischer Beobachter:

\begin{equation}
    \dot{\hat{v}}_i = S_w \hat{v}_i + L_i a_{i0} (w-\hat{w}_i)
\end{equation}
(Die Beobachterrückführmatrizen $L_i$ lassen sich für jeden Agenten
unterschiedlich wählen.)


Hier wird ein lokaler Führungsbeobachter entworfen.
Dafür betrachtet man den globalen Beobachtungsfehler $e_i = v_0 - \hat{v}_i$:
\begin{equation}
    \dot{e}_i = (S_w - L_i a_{i0} Q_w) e_i
\end{equation}
Daraus dass $(Q_w, S_w)$ beobachtbar ist, kann immer ein passendes $L_i$ bestimmt werden,
um die Beobachterfehlerdynamik zu stabilisieren.

Das kann z.B mittels Eigenwertvorgabe für $S_w - L_i a_{i0} Q_w$ und
Koeffizientenvergleich gemacht werden.

\subsubsection{Nichtinformierte Agenten}

Bei nichtinformierten Agenten wird ein kooperativer Führungsbeobachter konstruiert,
der dann durch diffusive Kopplung mit den anderen Agenten kommuniziert:

\begin{equation}
    \dot{\hat{v}}_i = S_w \hat{v}_i + L \sum_{j=1}^N a_{ij} (\hat{w}_j - \hat{w}_i)
\end{equation}
(Gemeinsame Beobachterrückführmatrix $L$.)

Hierfür wird der kooperative Führungsbeobachter entworfen.
Dafür betrachtet man den globalen Beobachtungsfehler $e_i = v_0 - \hat{v}_i$:
\begin{align}
    \dot{e}_i &= S_w e_i - \sum_{j=1}^N a_{ij} L Q_w (\hat{v}_j - \hat{v}_i) \\
    &= S_w e_i - \sum_{j=1}^N a_{ij} L Q_w (e_i - e_j)
\end{align}

Zur Bestimmung von $L$ bildet man die Laplacematrix des erweiterten Graphen in Blockmatrixform:

\begin{equation}
    L_{\overbar{\mathcal{G}}} = \begin{bmatrix}
        0 & 0_{1\times M} & 0_{1 \times N-M}\\
        L_{21} & L_{22} & 0_{M \times N-M} \\
        0_{N-M \times 1} & L_{32} & \bm{L_{33}}
    \end{bmatrix}
\end{equation}

Dabei ist $M$ die Anzahl informierter Agenten, $N$ die Anzahl Agenten, und $N-M$ die
Anzahl Nichtinformierter Agenten.
Achtung: Das erfordert im Allgemeinen zuerst ein Umsortieren der Agenten, sodass
an Stelle 0 der Führungsagent, dann die informierten Agenten und dann die
nichtinformierten Agenten in der Reihenfolge vorkommen.

\begin{itemize}
    \item Nullzeile: Keine Verbindungen zum Führungsagenten
    \item $L_{21}$: Verbindungen Führungsagent $\rightarrow$ Informierte Agenten
    \item $L_{22}$: Verbindungen zwischen informierten Agenten
    \item $0_{M \times N-M}$: Keine Verbindungen Nichtinformierte Agenten $\rightarrow$ Informierte Agenten
    \item $0_{N-M \times 1}$: Keine Verbindungen Führungsagent $\rightarrow$ Nichtinformierte Agenten
    \item $L_{32}$: Verbindungen Informierte Agenten $\rightarrow$ Nichtinformierte Agenten
    \item $\bm{L_{33}}$: Verbindungen zwischen nichtinformierten Agenten
\end{itemize}

Die Rückführmatrix $L$ bestimmt sich dann als
\begin{equation}
    L= \bar{c} \bar{P} Q_w^T \bar{R}^{-1}
\end{equation}
mit
\begin{equation}
    \bar{c} \geq \frac{1}{2 \min_{i}\operatorname{Re} \lambda_i(L_{33})}
\end{equation}
und $\bar{P}$ (symmetrisch) als Lösung der ARE
\begin{equation}
    S_w \bar{P} + \bar{P} S_w^T - \bar{P} Q_w^T \bar{R}^{-1} Q_w \bar{P} + \bar{Q} = 0
\end{equation}

Wenn $\overbar{\mathcal{G}}$ den Führungsagenten als Wurzelknoten enthält.

\subsubsection{Fehlerzustandsrückführung mit Führungsgrößenaufschaltung}
\label{sec:heterogen_regler}

Dann wird eine Führungsgrößenaufschaltung konstruiert, die eine Zustandsrückführung $K$
der Agentenzustände und eine Aufschaltung $M$ des rekonstruieren
Führungsagentenzustands enthält:

\begin{align}
    u_i &= \underbrace{-K_i (x_i - \Pi_i v_0)}_\text{Folgeregler}
    + \underbrace{\Gamma_i v_0}_\text{Steuerung} \\
    &= -K_i x_i + (K_i \Pi_i + \Gamma_i) v_0 \\
    &= -K_i x_i + M_i v_0
\end{align}

Dabei muss $A_i - B_i K_i$ Hurwitz Matrix sein (z.B. Eigenwertvorgabe) und es muss
gelten:
\begin{align}
    \Pi_i S_w - A_i \Pi_i &= B_i \Gamma_i \\
    C_i \Pi_i &= Q_w
\end{align}

Für die eindeutige Lösbarkeit davon ist Voraussetzung, dass keine "invariante Nullstelle"
von $(C_i, A_i, B_i)$ mit einem Eigenwert von $S_w$ übereinstimmt:
Für alle Eigenwerte $s$ von $S_w$ muss gelten:

\begin{equation}
    \det \begin{bmatrix}
        A_i -sI & B_i \\
        C_i & 0
    \end{bmatrix}
    \neq 0
\end{equation}

\pagebreak
\section{Schaltende Netze}

Schaltende Netzte modellieren wir dadurch dass wir nicht den konstanten
Graphen $\mathcal{G}$ betrachten sondern den Graphen $\mathcal{G}_{\sigma(t)}$,
wobei $\sigma(t)$ zu jedem Zeitpunkt den momentan aktiven Graphen
aus einer endlichen Menge an Graphen auswählt.

Hier schränken wir $\sigma(t)$ noch soweit ein dass ein Graph mindestens
für die Verweildauer $\tau$ aktiv bleibt, in dieser Zeit ändert sich
$\sigma(t)$ also nicht.
Das Schaltsignal $\sigma(t)$ wird als bekannt und deterministisch 
angenommen.

Der \emph{Vereinigungsgraph} $\tilde{\mathcal{G}}$ beinhaltet die Kanten
aller Graphen in der Graphenmenge.
Die Kantengewichte werden dabei addiert.

Zur Synchronisierung wird ein kooperativer Führungsbeobachter genutzt,
das Problem wird auf die leader-follower Zustandssynchronisierung
des homogenen Führungsmodells zurückgeführt.

\begin{equation}
    \dot{\hat{v}}_i = S_w \hat{v}_i + \mu \left( \sum_{j=1}^N a_{ij}(t)
    (\hat{v}_j - \hat{v}_i) + a_{i0}(t)(v_0 - \hat{v}_i) \right)   
\end{equation}

Hier wird eine gemeinsame Verstärkung $\mu >0$ gewählt und informierte
und Nichtinformierte Agenten gemeinsam betrachtet.

Für den Folgefehler $e_i = \hat{v}_i - v_0$ ergibt sich nach Aggregation
\begin{equation}
    \dot{e} = (I_N \otimes S_w - \mu H_{\sigma(t)} \otimes I_{n_w}) e
\end{equation}

Die Folgefehlerdynamik ist stabil für alle $\mu > 0$ wenn der Vereinigungsgraph
mit Agent 0 als einzigem Wurzelknoten zusammenhängend ist und
$\operatorname{Re}\lambda(S_w) \leq 0$.

Für die Bestimmung von $\mu$ kann z.B. Eigenwertvorgabe für die
Beobachterfehlerdynamik gemacht werden, sodass eine bestimmte
Stabilitätsreserve eingehalten wird.
In der Übung wurde hier dann das Statische Kommunikationsnetz
mit $\tilde{H}$ betrachtet.

Die Eigenwerte sollten über die Schur-Zerlegung von $\tilde{H}$
bestimmt werden:
\begin{equation}
    T^{-1}\tilde{H}T = \begin{bmatrix}
        \lambda_1(\tilde{H}) & &* \\
        & \ddots & \\
        0 & & \lambda_N(\tilde{H})
    \end{bmatrix}=Q
\end{equation}

Dafür wird eine Ähnlichkeitstrafo mit $(T\otimes I_{n_w})$ gemacht:

\begin{align}
    &(T^{-1} \otimes I_{n_w})(I_N \otimes S_w - \mu \tilde{H} \otimes I_{n_w})(T\otimes I_{n_w}) \\
    &= I_N \otimes S_w - \mu Q \otimes I_{n_w} \\
    &= \begin{bmatrix}
        S_w - \mu \lambda_1(\tilde{H}) I_{n_w} & &*\\
        & \ddots & \\
        0 & & S_w - \mu \lambda_N(\tilde{H}) I_{n_w}
    \end{bmatrix}
\end{align}

Mit ein paar Rechenregeln für Determinanten und scharfem Nachdenken stellt man
nun fest dass die kombinierten Eigenwerte der Diagonalelemente der Eigenwerte
der ursprünglichen Beobachterfehlerdynamik entsprechen.

Das eben bestimmte $\mu$ kann nun auch für den schaltenden Graphen verwendet
werden. Die Stabilitätsreserve wird aber nicht unbedingt eingehalten, da bei der
Bestimmung eine abweichende Topologie genutzt wurde.
Exponentialle Stabilität und positive Stabilitätsreserve ist aber dennoch auch
für den schaltenden Graphen garantiert.
